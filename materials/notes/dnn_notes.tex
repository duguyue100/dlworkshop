\documentclass[twoside,11pt]{article}

% Any additional packages needed should be included after jmlr2e.
% Note that jmlr2e.sty includes epsfig, amssymb, natbib and graphicx,
% and defines many common macros, such as 'proof' and 'example'.
%
% It also sets the bibliographystyle to plainnat; for more information on
% natbib citation styles, see the natbib documentation, a copy of which
% is archived at http://www.jmlr.org/format/natbib.pdf

\usepackage{jmlr2ed}

% Definitions of handy macros can go here

\newcommand{\dataset}{{\cal D}}
\newcommand{\fracpartial}[2]{\frac{\partial #1}{\partial  #2}}

% Heading arguments are {volume}{year}{pages}{submitted}{published}{author-full-names}

%\jmlrheading{1}{2000}{1-48}{4/00}{10/00}{Marina Meil\u{a} and Michael I. Jordan}

% Short headings should be running head and authors last names

\ShortHeadings{Introduction to Deep Neural Networks: GPU computing perspective}{Hu and Loo}
\firstpageno{1}

\begin{document}

\title{Introduction to Deep Neural Networks \\
\normalsize GPU computing perspective}

\author{\name Yuhuang Hu \email duguyue100@gmail.com \\
       \name Chu Kiong Loo \email ckloo.um@um.edu.my \\
       \addr Advanced Robotic Lab\\
       Department of Artificial Intelligence\\
       Faculty of Computer Science \& IT \\
       University of Malaya}
       
%\editor{}

\maketitle

\begin{abstract}%   <- trailing '%' for backward compatibility of .sty file
Deep Learning (DL) might be \emph{the} hottest topic right now in Machine Learning and Artificial Intelligence community. The successful stories of DL in weak AI problems are all over the news (especially in recent 2 years). It is critical to understand Deep Learning concepts while one is building up state-of-art intelligent software. This note is designed to help 
\end{abstract}

\begin{keywords}
  Deep Learning, Deep Neural Networks, GPU computing
\end{keywords}

\newpage
\section{Getting Started}

\subsection{Basics in machine learning}

\subsection{Preprocessing}

\subsection{Gradient-based optimization}

\newpage
\section{Classification}

\subsection{What is classification?}

\subsection{Logistic regression and softmax regression}

\subsection{Linear support vector machine}

\newpage
\section{Multilayer Perceptron}

\newpage
\section{Auto-encoders}



\newpage
\section{Convolutional Neural Networks}

\emph{Convolutional Neural Networks} (CNNs/ConvNets) is a variation of MLP networks.

\newpage
\section{Long-short Term Memory}

\newpage
\section{General Discussion}

\newpage

% Acknowledgements should go at the end, before appendices and references

\acks{We would like to acknowledge support for this project
from the National Science Foundation (NSF grant IIS-9988642)
and the Multidisciplinary Research Program of the Department
of Defense (MURI N00014-00-1-0637). \cite{chow:68} }

% Manual newpage inserted to improve layout of sample file - not
% needed in general before appendices/bibliography.

\newpage

%\appendix

\vskip 0.2in
\bibliography{dnnotesref}

\end{document}
