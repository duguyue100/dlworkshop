\documentclass{beamer}

\usepackage{amsmath, amssymb}
\usepackage{graphicx}
\usepackage{url}
\usepackage{xspace}
\usepackage{pifont}
\usepackage{minted}
\usepackage{verbatim}
\usepackage{wasysym}
\usepackage[numberedbib]{apacite}
\usepackage{bm}

\DeclareMathOperator{\Tr}{Tr}

\usetheme{AnnArbor}
\usefonttheme[onlymath]{serif}

\title[Intro DNNs]{\textbf{Practical Deep Neural Networks} \\
\textbf{\normalsize GPU computing perspective}\\
\normalsize Theano Tutorial}
\author{Yuhuang Hu \and Chu Kiong Loo}
\institute[UM]{Advanced Robotic Lab\\
Department of Artificial Intelligence\\
Faculty of Computer Science \& IT\\
University of Malaya}

\date{}

\begin{document}

\frame{\titlepage}

\begin{frame}
  \frametitle{Outline}

  \tableofcontents
\end{frame}

\AtBeginSection[]
  {
     \begin{frame}
     \frametitle{Outline}
     \tableofcontents[currentsection]
     \end{frame}
  }

\section{Introduction}

\begin{frame}
  \frametitle{Prerequisites}

  \begin{itemize}
    \item[\checkmark] A machine that installed Python and working Theano library.
    \item[\checkmark] Know basic Python
    \item[\checkmark] Know basic numpy
  \end{itemize}
\end{frame}

\begin{frame}
  \frametitle{Suggest Readings}
  
  \begin{itemize}
    \item[\ding{45}] Python numpy tutorial

      \url{http://cs231n.github.io/python-numpy-tutorial/}

    \item[\ding{45}] Theano tutorial

      \url{http://deeplearning.net/software/theano/index.html}
  \end{itemize}
\end{frame}

\section{Baby Steps}

\begin{frame}[fragile]
  \frametitle{Adding two scalars}

\begin{minted}{python}
import theano.tensor as T
from theano import function

x = T.dscalar('x')
print x.type
y = T.dscalar('y')
z = x + y
f = function([x, y], z) # The function!
\end{minted}
And now that we’ve created our function we can use it:

\begin{minted}{python}
print f(2,3);
print f(16.3, 12.1);
\end{minted}
\end{frame}

\begin{frame}[fragile]
  \frametitle{Adding two matrices}

\begin{minted}{python}
x = T.dmatrix('x')
y = T.dmatrix('y')
z = x + y
f = function([x, y], z)

print f([[1, 2], [3, 4]], [[10, 20], [30, 40]])

import numpy
print f(numpy.array([[1, 2], [3, 4]]),
        numpy.array([[10, 20], [30, 40]]))
\end{minted}
\end{frame}

\begin{frame}
  \Large\centering
  Theano is a COMPILER!
\end{frame}

\section{Complex Examples}

\begin{frame}[fragile]

\frametitle{Logistic Function}
\begin{equation*}
  f(x)=\frac{1}{1+\exp(-x)}
\end{equation*}

\begin{minted}[frame=lines]{python}
x = T.dmatrix('x')
s = 1 / (1 + T.exp(-x))
logistic = function([x], s)
print logistic([[0, 1], [-1, -2]])
\end{minted}
\end{frame}

\begin{frame}[fragile]
  \frametitle{Compute more than one thing}
\begin{minted}[frame=lines]{python}
a, b = T.dmatrices('a', 'b')
diff = a - b
abs_diff = abs(diff)
diff_squared = diff**2
f = function([a, b], [diff, abs_diff, diff_squared])

print f([[1, 1], [1, 1]], [[0, 1], [2, 3]])
\end{minted}
\end{frame}

\begin{frame}[fragile]
  \frametitle{set default value for argument}

\begin{minted}[frame=lines]{python}
from theano import Param
x, y = T.dscalars('x', 'y')
z = x + y
f = function([x, Param(y, default=1)], z)

print f(33)
print f(33, 2)
\end{minted}
\end{frame}

\begin{frame}[fragile]
  \frametitle{Using Shared Variables}

\begin{minted}[frame=lines, fontsize=\footnotesize]{python}
from theano import shared
state = shared(0)
inc = T.iscalar('inc')
accumulator = function([inc], state, updates=[(state, state+inc)])

# get value
print state.get_value()
accumulator(1)
print state.get_value()
accumulator(300)
print state.get_value()

# set value
state.set_value(-1)
decrementor = function([inc], state, updates=[(state, state-inc)])
decrementor(2)
print state.get_value()
\end{minted}
\end{frame}

\begin{frame}[fragile]
  \frametitle{Using Shared Variable}

\begin{minted}[frame=lines, fontsize=\footnotesize]{python}
fn_of_state = state * 2 + inc

# The type of foo must match the shared variable we are replacing
# with the ``givens``
foo = T.scalar(dtype=state.dtype)
skip_shared = function([inc, foo], fn_of_state,
                       givens=[(state, foo)])

print skip_shared(1, 3)
print state.get_value()
\end{minted}
\end{frame}

\section{Derivatives in Theano}

\begin{frame}[fragile]
  \frametitle{Computing Gradients}

\begin{minted}[frame=lines, fontsize=\footnotesize]{python}
from theano import pp
x = T.dscalar('x')
y = x ** 2
gy = T.grad(y, x)
pp(gy)  # print out the gradient prior to optimization
f = function([x], gy)
print f(4)
print f(94.2)
\end{minted}
\end{frame}

\section{Conditions and Loop}

\begin{frame}[fragile]
  \frametitle{Conditions}
\begin{minted}[frame=lines, fontsize=\footnotesize]{python}
from theano import tensor as T
from theano.ifelse import ifelse
import theano, time, numpy

a,b = T.scalars('a', 'b')
x,y = T.matrices('x', 'y')

z_switch = T.switch(T.lt(a, b), T.mean(x), T.mean(y))
z_lazy = ifelse(T.lt(a, b), T.mean(x), T.mean(y))

f_switch = theano.function([a, b, x, y], z_switch,
                    mode=theano.Mode(linker='vm'))
f_lazyifelse = theano.function([a, b, x, y], z_lazy,
                    mode=theano.Mode(linker='vm'))
\end{minted}
\end{frame}

\begin{frame}[fragile]
  \frametitle{Scan}

\begin{minted}[frame=lines, fontsize=\scriptsize]{python}
import theano
import theano.tensor as T
import numpy as np
# defining the tensor variables
X = T.matrix("X")
W = T.matrix("W")
b_sym = T.vector("b_sym")

results, updates = theano.scan(lambda v: T.tanh(T.dot(v, W) + b_sym),
                               sequences=X)
compute_elementwise = theano.function(inputs=[X, W, b_sym],
                                      outputs=[results])
# test values
x = np.eye(2, dtype=theano.config.floatX)
w = np.ones((2, 2), dtype=theano.config.floatX)
b = np.ones((2), dtype=theano.config.floatX)
b[1] = 2
print compute_elementwise(x, w, b)[0]
# comparison with numpy
print np.tanh(x.dot(w) + b)
\end{minted}
\end{frame}

\section*{Q\&A}

\begin{frame}
  \frametitle{Q\&A}

  \begin{figure}
    \centering
    \includegraphics[width=0.4\textwidth]{theanoqanda.jpg}
  \end{figure}
\end{frame}

\end{document}
%%% Local Variables:
%%% mode: latex
%%% TeX-master: t
%%% End:
